\documentclass{article}
\ifx\RunningPreview\undefined
\usepackage{hyperref}
\fi
\usepackage[fleqn]{amsmath}
\usepackage{amsfonts}
\usepackage{float}
\usepackage[DIV15]{typearea}
\usepackage[utf8]{inputenc}
\usepackage[ngerman]{babel}
\usepackage{pstricks,pst-plot,pst-node,pst-circ}
%%%%%%%%%%%%%%%%%%%%%%%%%%%%%%%%%%%%%%%%%%%%%%%%%%%%%%%%%%%%%%%%%%%%%% 
\let\l\left\let\r\right\let\cs\csname\let\ecs\endcsname\let\ea\expandafter
\let\eps\varepsilon
\def\mdo#1{{\def\tmp{#1}\def\next{\relax}\ifx\tmp\next\else\def\next{\do{#1}\mdo}\fi\ea}\next}
%%%%%%%%%% 
\def\ddx#1{\frac{d#1}{dx}}
\def\DDx#1{\frac{\Delta#1}{\Delta x}}
\def\ddt#1{\frac{d#1}{dt}}
\def\pdx#1{\frac{\partial #1}{\partial x}}
\def\pdt#1{\frac{\partial #1}{\partial t}}
\def\uuline#1{\underline{\underline{#1}}}
%%%%%%%%%%%%%%%%%%%%%%%%%%%%%%%%%%%%%%%%%%%%%%%%%%%%%%%%%%%%%%%%%%%%%% 
%% pst setup
\def\euler{2.71828 }
\psset{gridcolor=green,subgridcolor=yellow,gridwidth=0.1\pslinewidth,subgridwidth=0pt}
\definecolor{lightblue}{rgb}{0.9 0.9 1.0}
\SpecialCoor
%%%%%%%%%%%%%%%%%%%%%%%%%%%%%%%%%%%%%%%%%%%%%%%%%%%%%%%%%%%%%%%%%%%%%% 
\newtheorem{Def}{Definition}
%%%%%%%%%%%%%%%%%%%%%%%%%%%%%%%%%%%%%%%%%%%%%%%%%%%%%%%%%%%%%%%%%%%%%% 
\title{Differenzialquotient}
\begin{document}
\maketitle
\tableofcontents
\section{Einleitung}
\label{sec:intro}
Die \emph{Sekante}, die eine Kurve $f(x)$ an zwei vorgegebenen Stellen $x_1,x_2$ schneidet, ist die Gerade
\begin{align}
  g(x) &:= f(x_1) + \frac{f(x_2)-f(x_1)}{x_2-x_1}\cdot(x-x_1).\label{eq:intro:secant}
\end{align}
Als \emph{Tangente} an der Stelle $x_1$ wird diejenige Gerade
bezeichnet, die sich als Grenzgerade bei Annäherung von $x_2$ an
$x_1$ ergibt. Im Rahmen der Einleitung wollen wir uns mit dieser etwas
vagen Andeutung und dem folgenden Beispiel zufrieden geben. In
Abschnitt~\ref{sec:limit} gehen wir dann näher darauf ein, wie die
Annäherung von $x_2$ an $x_1$ gemeint ist.

\begin{figure}[H]
  \psset{unit=1.5cm}
  \begin{pspicture}(-2,-4)(9,5)
    % \psgrid
    %%%%%%%%%%%%%%%%%%%%%%%%%%%%%%%%%%% 
    % General setup
    \def\mypic#1{
      \psaxes[Dx=0.5,Dy=0.5]{->}(0,0)(-1,-3)(2.5,4.5)[$x$,270][$y$,180]
      \psplot{-1}{2}{ x x mul}
      \rput(0.5,-3.5){#1}
    }
    %%%%%%%%%%%%%%%%%%%%%%%%%%%%%%%%%%% 
    \mypic{{\color{blue}Differenz\textbf{en}quotient}}
    % Anstieg: 2,
    % Sekante:
    % y = 2*x - 0.75
    % xl = -1, yl = -2.75
    % xr = 2, yr = 3.25
    \psline[linecolor=blue](-1,-2.75)(2,3.25) % Sekante
    {
      % Maßlinien
      \psset{linewidth=0.5\pslinewidth}
      \psline(0.5,-1)(0.5,0.25)\rput[t](0.5,-1.1){$x_1$}
      \psline(1.5,-1)(1.5,2.25)\rput[t](1.5,-1.1){$x_2$}
      \pcline{->}(0.5,-0.8)(1.5,-0.8)\Aput{$\Delta x$}
      \psline(0.5,0.25)(2.3,0.25)\rput[l](2.4,0.25){$y_1$}
      \psline(1.5,2.25)(2.3,2.25)\rput[l](2.4,2.25){$y_2$}
      \pcline{->}(2,0.25)(2,2.25)\Bput{$\Delta y$}
    }
    %%%%%%%%%%%%%%%%%%%%%%%%%%%%%%%%%%% 
    \rput(5,0){
      \mypic{{\color{red}Differenz\textbf{ial}quotient}}
      % Anstieg: 2*0.5 = 1
      % Tangente:
      % xl = -1, yl = -1.25
      % xr = 2, yr = 1.75
      \psline[linecolor=red](-1,-1.25)(2,1.75) % Tangente
      {
        \psset{linewidth=0.5\pslinewidth}
        \psline(0.5,-1)(0.5,0.25)\rput[t](0.5,-1.1){$x_1$}
        \psline(1.5,-1)(1.5,1.25)\rput[t](1.5,-1.1){$x_1+\Delta x$}
        \pcline{->}(0.5,-0.8)(1.5,-0.8)\Aput{$\Delta x$}
        \psline(0.5,0.25)(2.3,0.25)\rput[l](2.4,0.25){$y_1$}
        \psline(1.5,1.25)(2.3,1.25)
        \pcline{->}(2,0.25)(2,1.25)\Bput{$\Delta y = \ddx{f}(x_1)\cdot \Delta x$}
      }
    }
    %%%%%%%%%%%%%%%%%%%%%%%%%%%%%%%%%%% 
  \end{pspicture}
  \caption{Differenzenquotient als Anstieg der Sekanten und Differenzialquotient als Anstieg der Tangenten}
  \label{fig:differenceQuotientAndDerivative}
\end{figure}

In Abbildung~\ref{fig:differenceQuotientAndDerivative} ist zweimal die quadratische Funktion
\begin{align*}
  y&=f(x) := x^2
\end{align*}
dargestellt.
Die Sekante, die den Graph von $f$ an den Stellen
$x_1=0.5$ und $x_2=1.5$ schneidet, ist blau eingezeichnet und rot die
Tangente an der Stelle $x_1$.

Der Anstieg der Sekante ergibt sich aus dem Differenzenquotient
\begin{align*}
  \frac{\Delta f}{\Delta x}(x_1,x_2) &= \frac{f(x_2)-f(x_1)}{x_2-x_1}
\end{align*}
Lässt man die Differenz $\Delta x=x_2-x_1$ betragsmäßig immer kleiner
werden, nähert sich der Differenzenquotient immer mehr dem Anstieg der
Tangenten, der als \emph{Differenzialquotient} $\ddx{f}(x_1)$ an der
Stelle $x_1$ bezeichnet wird.

Gleichungen mit Differenzialquotienten heißen \emph{Differenzialgleichungen}.

Interpretiert man die x-Achse als Zeit und die y-Achse als Weg, so ist
der Differenzenquotient $\frac{y_2-y_1}{x_2-x_1}$ die
Durchschnittsgeschwindigkeit im Zeitintervall von $x_1$ bis $x_2$ und
der Differenzialquotient $\ddx f(x_1)$ ist die Momentangeschwindigkeit
zum Zeitpunkt $x_1$.

Fast alle physikalischen Erscheinungen lassen sich durch Differenzialgleichungen beschreiben. In den folgenden Unterabschnitten wollen wir uns einige wenige Beispiele anschauen.
\subsection{Mechanik}
Mit der Federkonstante $k$ und der entspannten Länge $l$ einer Feder ergibt sich die Federkraft aus der Gleichung
\begin{align*}
  F(x) &= k(l-x).
\end{align*}
Die Beschleunigung $a(t)$ eines Körpers mit der Masse $m$ zur Zeit $t$ ergibt sich mit der auf den Körper momentan einwirkenden Kraft $F(t)$ aus der Gleichung
\begin{align*}
  m\cdot a(t) &=F(t).
\end{align*}
Die Beschleunigung ist dabei als momentane Änderung der Geschwindigkeit $v(t)$, also als Differenzialquotient
\begin{align*}
  a(t) &= \ddt{v}(t)
\end{align*}
zu interpretieren und die Geschwindigkeit ist die momentane Änderung
\begin{align*}
  v(t) &= \ddt{x}(t)
\end{align*}
des Ortes $x(t)$. So ergibt sich für die Kombination des Körpers mit der Feder das Differenzialgleichungssystem
\begin{align*}
  m\cdot \ddt v(t) &= k(l-x(t)),\\
  \ddt x(t) &= v(t).
\end{align*}
\subsection{Thermodynamik}
Wir betrachten eine sehr lange Metallstange die wir mit einer
Ortsvariable $x$ koordinatisieren. Um das Beispiel einfach zu halten
interessieren wir uns erst einmal nicht für die Enden der Stange und
nehmen an, dass $x$ von $-\infty$ bis $+\infty$ läuft.

Die Stange habe eine von der Koordinate $x$ unabhängige
Querschnittsfläche $A$, eine Dichte $\rho$ und eine Wärmekapazität
$c$.

Wir nehmen an, dass zum Anfangszeitpunkt $t=0$ der Temperaturverlauf
$T_0(x)$ gegeben ist und interessieren uns zu jedem Zeitpunkt $t> 0$
für den Temperaturverlauf $T(x,t)$ an allen Stellen $x$ der Stange.

Der \emph{Wärmestrom} $\dot q(x,t)$ ist die Wärme, die pro Zeiteinheit
durch die Querschnittsfläche $A$ an der Stelle $x$ in positiver
$x$-Richtung strömt.

Die Wärme fließt immer von Punkten höherer Temperatur zu Punkten
niedrigerer Temperatur. Je höher die Temperaturdifferenz ist, desto
größer ist der Wärmestrom.

\begin{figure}[H]
  \centering
  \begin{pspicture}(-4,-2)(4,5)
    % \psgrid
    \definecolor{lightgrey}{rgb}{0.95 0.95 0.95}
    \psline[linewidth=3\pslinewidth](-2.5,-1)(2.5,-1)
    \psline[linewidth=3\pslinewidth](-2.5,1)(2.5,1)
    \psframe[linestyle=none,fillstyle=solid,fillcolor=lightgrey](-2.5,-1)(2.5,1)
    \psline{->}(-3,0)(3,0)
    \rput[tl](3,0){$x$}
    \psline(-1.5,-1)(-1.5,1)\psline[linestyle=dashed](-1.5,1)(-1.5,4.5)
    \rput[tr](-1.5,-0.1){$x_1$}
    \psline(1.5,-1)(1.5,1)\psline[linestyle=dashed](1.5,1)(1.5,4.5)
    \rput[tr](1.5,-0.1){$x_2$}
    \psline{->}(-1.5,0.2)(-1,0.2)\rput[bl](-1,0.2){$\dot q(x_1,t)$}
    \psline{->}(1.5,0.2)(2,0.2)\rput[bl](2,0.2){$\dot q(x_2,t)$}
    \def\myshift{1.2 }
    \def\myscale{1.5 }
    \def\fun{ \myshift sub \myscale div dup mul neg \euler exch exp }
    \rput(0,3){
      \psaxes{->}(0,0)(-3.5,-1.5)(3.5,1.5)[$x$,270][$T$,180]
      \psplot{-3}{3}{x \fun}
      \rput(!-1.5 -1.5 \fun){\psline[linecolor=blue](0,0)(1;!-1.4\fun-1.5\fun sub 0.1 atan)}
      \rput(!1.5 1.5 \fun){\psline[linecolor=red](0,0)(1;!1.6 \fun 1.5 \fun sub 0.1 atan)}      
      \rput[br](-1.6,0.2){$\pdx T(x_1,t)$}
      \rput[bl](1.6,1){$\pdx T(x_2,t)$}
    }
  \end{pspicture}
  \caption{Wärmeleitung in einer unendlich langen Metallstange; Unten: Ausschnitt der Metallstange; Oben: Temperaturverlauf entlang des Ausschnitts zusammen mit Tangentenausschnitten für den Temperaturanstieg $\pdx{T}(x,t)$ an den Stellen $x=x_1$ und $x=x_2$}
  \label{fig:heatFlow}
\end{figure}
Physikalische Messungen zeigen, dass der Wärmestrom direkt
proportional zum Temperaturgefälle $-\pdx T(x,t)$ ist. Der
Differenzialquotient mit dem geschwungenen $\partial$ heißt
\emph{partielle Ableitung} und bedeutet, dass alle Größen, nach denen
nicht differenziert wird, während des Differenzierens konstant zu
halten sind.  Die Proportionalitätskonstante $k$ ist nur vom Material
und von der Querschnittsfläche abhängig. Wir erhalten also für den
Wärmefluss die Gleichung
\begin{align}
  \dot q(x,t) &= -k\cdot \pdx T(x,t).\label{eq:thermDyn:heatCond}
\end{align}
Betrachten wir ein kurzes Stück der Stange, das von $x_1$ bis $x_2$
geht (mit $x_1 < x_2$, siehe Abbildung~\ref{fig:heatFlow}). Die Länge
dieses Stücks ist $\Delta x:=x_2-x_1$.

Für eine Erhöhung der Temperatur $\bar T(x_1,x_2,t)$ des Stückchens um
$\Delta\bar T$ wird die Wärmemenge
\begin{align*}
  \Delta Q&= \underbrace{A\cdot \Delta x}_{\text{\phantom{V}\pnode{V}}} \cdot \rho\cdot c\cdot \Delta \bar T(x_1,x_2,t)
            \rput(V){\text{\small Volumen}}
\end{align*}
benötigt. Wie in Abbildung~\ref{fig:heatFlow} ersichtlich ist, fließt
der Wärmestrom $\dot q(x_1,t)$ in den Abschnitt $[x_1,x_2]$ hinein,
während $\dot q(x_2,t)$ aus dem Abschnitt herausfließt. Für eine über
das Stückchen gemittelte momentane Temperaturänderung
$\pdt{\bar T}(x_1,x_2,t)$ ist ein Wärmestrom
\begin{align*}
  \dot q(x_1,t) - \dot q(x_2,t) &= A\cdot \Delta x\cdot \rho\cdot c\cdot \pdt{\bar T}(x_1,x_2,t)
\end{align*}
erforderlich. Division durch $\Delta x$ und Berücksichtigung von
$x_2=x_1+\Delta x$ liefert
\begin{align*}
  -\frac{\dot q(x_1+\Delta x,t) - \dot q(x_1,t)}{\Delta x} &= A\cdot \rho\cdot c\cdot \pdt{\bar T}(x_1,x_2,t).
\end{align*}
Auf der linken Seite taucht der Differenzenquotient des Wärmeflusses
bzgl. $x$ auf.  Lassen wir noch $\Delta x$ gegen null streben, so
erhalten wir
\begin{align}
  -\pdx{\dot q}(x_1,t) &= A\cdot \rho \cdot c\cdot \pdt{T}(x_1,t).\label{eq:thermDyn:heatStorage}
\end{align}
Dabei haben wir berücksichtigt, dass die über das Intervall
$[x_1,x_2]$ gemittelte Temperatur $\bar T(x_1,x_2,t)$ gegen die
Temperatur $T(x_1,t)$ an der Stelle $x_1$ strebt, wenn $x_2$ gegen
$x_1$ strebt.

Da wir in der Formel nicht mehr zwischen den zwei Koordinaten
$x_1,x_2$ unterscheiden müssen ($x_2$ strebt ja gegen $x_1$), können
wir auch einfach wieder $x$ schreiben und die zwei Gleichungen
\eqref{eq:thermDyn:heatCond} und \eqref{eq:thermDyn:heatStorage} zum
folgenden System zusammenfassen:
\begin{align*}
  \dot q(x,t) &= -k \pdx{T}(x,t),\\
  -\pdx{\dot q}(x,t) &= A\cdot \rho \cdot c\cdot \pdt{T}(x,t).
\end{align*}
Dieses (partielle) Differenzialgleichungssystem beschreibt die
Wärmeleitung in der Metallstange. Um dir noch einen Anfasser für eine
eventuelle Literaturrecherche zu geben sei hier ohne weitere
Erläuterung erwähnt, dass die zwei partiellen Differenzialgleichungen
oft zur \emph{Wärmeleitungsgleichung}
$a\frac{\partial^2 T}{\partial x^2}(x,t) = \pdt{T}(x,t)$ mit
$a=\frac{k}{A\rho c}$ zusammengefasst werden.
\subsection{Elektrotechnik}
\label{sec:intro:et}
Die Spannung $U$ über einem Kondensator ist proportional zu der auf
ihm gespeicherten Ladung $Q$. Der Proportionalitätsfaktor ist die
Kapazität $C$ und die Gleichung für die Spannung am Kondensator lautet
\begin{align}
  C\cdot U &= Q
             \label{eq:electrics:cap:U}
\end{align}
Die momentane Erhöhung der Ladung~$\ddt Q(t)$ auf dem
Kondensator ist gleich dem Ladungsträgerzufluss, also dem Strom
$I(t)$, der am positiven Anschluss in den Kondensator hineinfließt:
\begin{align}
  I(t) &= \ddt Q(t)
         \label{eq:electrics:cap:I}
\end{align}
In Abschnitt~\ref{sec:ddx:linearity} sehen wir, dass konstante
Linearfaktoren bei der Differenzation einfach herausgezogen werden
können. So wird aus~\eqref{eq:electrics:cap:U} die Gleichung
\begin{align*}
  C\cdot \ddt U(t) &=\ddt Q(t)
\end{align*}
und mit~\eqref{eq:electrics:cap:I}
\begin{align}
  C\cdot \ddt U(t) &= I(t).
                     \label{eq:electricity:cap:UI}
\end{align}
Betrachten wir nun den Stromkreis in Abbildung~\ref{fig:electricity:cap} mit einer Spannungsquelle $B$ mit Quellspannung $U_B$, z.B. einer Batterie, einemWiderstand mit Widerstandswert $R$ und einem Kondensator mit Kapazität $C$.
\begin{figure}[H]
  \centering
  \begin{pspicture}(-1,-1)(4,5)
    \psset{tensionlabeloffset=1.5cm}
    % \psgrid
    \pnodes(0,0){gndA}(0,3){pSupply}(3,3){pC}(3,0){gndB}
    \battery[tension,tensionlabel=$U_B$](pSupply)(gndA){B}
    \resistor[tension,tensionlabel=$U_R$](pSupply)(pC){$R$}
    \capacitor[tension,tensionlabel=$U_C$,intensity,intensitylabel=$I$](pC)(gndB){$C$}
    \ncline{gndA}{gndB}
  \end{pspicture}
  \caption{Stromkreis aus Spannungsquelle, Widerstand und Kondensator}
  \label{fig:electricity:cap}
\end{figure}
Das Kirchhoffsche Stromgesetz gibt vor, dass durch alle drei Bauelemente der gleiche Strom, nämlich $I$ fließt.
Aus dem Kirchhoffschen Spannungsgesetz folgt die Maschengleichung
\begin{align*}
  0 &= -U_B + U_R(t) + U_C(t)
\end{align*}
Mit dem Ohmschen Gesetz $U_R = R\cdot I$ für den Widerstand erhält man daraus die Gleichung
\begin{align*}
  0 &= -U_B + R\cdot I(t) + U_C(t)\\
  I(t) &= \frac{U_B-U_C(t)}R
\end{align*}
Einsetzen dieser Gleichung in die Strom-Spannungsrelation~\eqref{eq:electricity:cap:UI} für den Kondensator liefert
\begin{align}
  C\ddt{U_C}(t) &= \frac{U_B-U_C(t)}{R}\notag\\
  \tau \ddt{U_C}(t) &= U_B-U_C(t)\label{eq:intro:et:dgl}
\end{align}
mit der \emph{Zeitkonstante} $\tau:=RC$. Die letzte Gleichung ist eine Differenzialgleichung für $U_C(t)$. Bei ihr hängt der Anstieg $\ddt {U_C}(t)$ vom Momentanwert $U_C(t)$ ab. Je weiter sich $U_C(t)$ dem Endwert $U_B$ nähert, desto kleiner wird der Anstieg (siehe Abbildung~\ref{fig:electricity:chargingCurve}).
\begin{figure}[H]
  \centering
  \psset{unit=2cm}
  \begin{pspicture}(-0.5,-0.5)(3.25,1.5)
    % \psgrid
    \psaxes{->}(0,0)(3.25,1.25)[$t/{\rm s}$,270][$({\color{red}U_C};{\color{green}U_B})/{\rm V}$,180]
    \psplot[linecolor=red,linewidth=2\pslinewidth]{0}{3}{1 \euler x neg exp sub}
    \psline[linecolor=green](0,1)(3,1)
    {
      \psset{linecolor=blue}
      \psline(0,0)(1,1)
      \psline(1,0)(1,1)
      \psline(0,0)(1,0)
      \def\myy{1 1 \euler div sub }
      \psline(! 1 \myy)(2,1)
      \psline(! 1 \myy)(! 2 \myy)
      \psline(! 2 \myy)(2,1)
      \def\myy{1 \euler -2 exp sub }
      \psline(! 2 \myy)(3,1)
      \psline(! 2 \myy)(! 3 \myy)
      \psline(! 3 \myy)(3,1)
    }
  \end{pspicture}
  \caption{Zeitverlauf von $U_C(t)$ (bei $\tau=1\rm s$ und $U_B=1\rm V$); In jedem Punkt von $U_C$ ist der Anstieg der Kurve so groß wie der Abstand des Momentanwertes vom Endwert $U_B$. Zur Verdeutlichung sind drei Anstiegsdreiecke blau eingezeichnet.}
  \label{fig:electricity:chargingCurve}
\end{figure}
\section{Grenzwerte}
\label{sec:limit}
\subsection{Differenzialquotient als Grenzwert von Differenzenquotienten}
\label{sec:limit:ddx}
Aus~\eqref{eq:intro:secant} kennen wir bereits die Gleichung für die
Sekante, die an den Stellen $x_1$ und $x_2$ eine Kurve $f(x)$
schneidet. Der Anstieg der Sekante ist
\begin{align}
  \frac{\Delta f}{\Delta x}(x_1,x_2) &= \frac{f(x_2)-f(x_1)}{x_2-x_1}
                                       \label{eq:limit:secant:slope}
\end{align}
Außerdem wurde in Abschnitt~\ref{sec:intro} erwähnt, dass die Tangente die Grenzgerade bei Annäherung von $x_2$ an $x_1$ ist.

Den Anstieg $\ddx f(x)$ der Tangente an der Stelle $x_1$ können wir jedoch nicht direkt aus Gleichung~\eqref{eq:limit:secant:slope} berechnen, da für $x_2=x_1$ der Nenner null wird.

Finden wir stattdessen eine Zahl $\ddx f(x_1)$, der sich der Sekantenanstieg~\eqref{eq:limit:secant:slope} annähert, wenn $x_2$ an $x_1$ heranrückt?
Wie ist dabei "`annähern"' zu verstehen?

Wir können den Differenzenquotienten~\eqref{eq:limit:secant:slope} nur
an Stellen $x_2$ ungleich $x_1$ auswerten und müssen deshalb bei jeder
konkreten Auswertung von $\frac{\Delta f}{\Delta x}(x_1,x_2)$ eine
absolute Abweichung
$|\ddx f(x_1) - \frac{\Delta f}{\Delta x}(x_1,x_2)|$ größer Null
zulassen.

\begin{Def}
  Wir definieren eine Zahl $\ddx f(x_1)$ als \emph{Grenzwert} von
  $\DDx f(x_1,x_2)$ bei Annäherung von $x_2$ an $x_1$, falls für jede
  (beliebig knapp über Null liegende) Fehlertoleranz $\eps>0$ die
  Abweichung $|\ddx f(x_1) - \frac{\Delta f}{\Delta x}(x_1,x_2)|$
  nicht größer als $\eps$ wird, wenn wir uns auf Stellen $x_2$
  ungleich $x_1$ beschränken, die einen (von $\eps$ abhängigen)
  Maximalabstand $\delta$ von $x_1$ haben.
\end{Def}

Als Beispiel ermitteln wir die Ableitung von $f(x)=x^2$ an der Stelle $x=x_1:=0.5$.

Für $x_2\neq x_1$ gilt
\begin{align}
  \DDx f(x_1,x_2) = \frac{x_2^2 - x_1^2}{x_2-x_1} = \frac{(x_2-x_1)(x_2+x_1)}{x_2-x_1}= x_2 + x_1
  \label{eq:limit:DDxf}
\end{align}
Im Zähler kam die binomische Formel $x_2^2-x_1^2=(x_2-x_1)(x_2+x_1)$ zum Einsatz.
Bei der Rechnung hebt sich danach der Nenner $x_2-x_1$ heraus. Die rechte
Seite ist frei von Divisionen, problemlos an der Stelle $x_2=x_1$
auswertbar und man erhält für sie an dieser Stelle $2 x_1$.

Wir vermuten also, dass
\begin{align}
  \ddx f(x_1) &= 2x_1
                \label{eq:limit:ddxf}
\end{align}
die Ableitung von $f(x)=x^2$ an der Stelle $x=x_1$ ist.

Wir prüfen ob $\ddx f(x_1)=2x_1$ der obigen Definition für den
Grenzwert von $\DDx f(x_1,x_2)$ bei Annäherung von $x_2$ an $x_1$ standhält.

Wir geben uns eine beliebig knapp über Null liegende Fehlertoleranz
$\eps>0$ vor und schauen, ob wir einen Maximalabstand $\delta>0$
finden, so dass für alle $x_2$, die keinen größeren Abstand von $x_1$
haben, die also $|x_2-x_1|\leq \delta$ erfüllen, die Relation
\begin{align*}
  \l|\ddx f(x_1) - \DDx f(x_1,x_2)\r| \leq \eps
\end{align*}
erfüllt ist. Einsetzen von \eqref{eq:limit:DDxf} und \eqref{eq:limit:ddxf} liefert die Ungleichung
\begin{align*}
  | \underbrace{2 x_1}_{\ddx f(x_1)} - \underbrace{(x_2 + x_1)}_{\DDx f(x_1,x_2)} | \leq \eps,
\end{align*}
die nach Vereinfachung in die folgende Relation übergeht:
\begin{align*}
  | x_1 - x_2 | \leq \eps
\end{align*}
Wie wir an dieser Ungleichung sehen, können wir im Beispiel $f(x)=x^2$
einfach den Maximalabstand $\delta = \eps$ nutzen. Somit ist die
Voraussetzung aus obiger Definition erfüllt und $\ddx f(x_1) = 2x_1$
ist der Grenzwert von $\DDx f(x_1,x_2)$ bei Annäherung von $x_2$ an
$x_1$.

In Abbildung~\ref{fig:limit:ddxf} sind zur Veranschaulichung bei
$f(x)=x^2$ die Grenzsekanten für die Fehlertoleranz $\eps=0.5$
eingetragen.  Bei dieser Fehlertoleranz ergibt sich ein möglicher
Maximalabstand $|x_2-x_1|=0.5$ und $x_2$ kann im Intervall von
$x_2=0$ bis $x_2=1$ gewählt werden.
\begin{figure}[H]
  \centering
  \psset{unit=1.5cm}
  \begin{pspicture}(-2,-4)(3,5)
    % \psgrid
    %%%%%%%%%%%%%%%%%%%%%%%%%%%%%%%%%%% 
    % General setup
    \pspolygon[linestyle=none,fillstyle=solid,fillcolor=lightblue](-1,-0.5)(2,1)(2,2.5)(-1,-2)
    \psaxes[Dx=0.5,Dy=0.5]{->}(0,0)(-1,-3)(2.5,4.5)[$x$,270][$y$,180]
    \psplot{-1}{2}{ x x mul}
    %%%%%%%%%%%%%%%%%%%%%%%%%%%%%%%%%%% 
    % Anstieg: 2*0.5 = 1
    % Tangente:
    % xl = -1, yl = -1.25
    % xr = 2, yr = 1.75
    \psline[linecolor=red](-1,-1.25)(2,1.75) % Tangente
    % Fehlergrenzen für Fehlertoleranz tol = 0.5
    % x = 0.5
    % y = 0.5^2 = 0.25
    % Anstieg Fehlergrenze: 0.5
    % Geradengleichung: y = 0.25 + 0.5*(x-0.5) = 0.5*x
    % xl = -1, yl = -0.5
    % xr = 2, yr = 1
    % Anstieg Fehlergrenze: 1.5
    % Geradengleichung: y = 0.25 + 1.5*(x-0.5) = -0.5 + 1.5*x
    % xl = -1, yl = -2
    % xr = 2, yr = 2.5
    \psline[linecolor=blue](-1,-0.5)(2,1)
    \psline[linecolor=blue](-1,-2)(2,2.5)
    %%%%%%%%%%%%%%%%%%%%%%%%%%%%%%%%%%% 
    \psset{linewidth=0.5\pslinewidth}
    \psline(1,0)(1,1)
    \psline(0.5,0)(0.5,0.25)
  \end{pspicture}
  \caption{Rot: Tangente für $f(x_1)=x_1^2$ im Punkt $x_1=0.5$; Blau: Bereich der Sekanten mit einer Fehlertoleranz von $\eps=0.5$ für die Abweichung des Differenzenquotienten $\DDx f(x_1,x_2)$ von der Tangente $\ddx f(x_1)$; Der Maximalabstand $|x_2-x_1|=0.5$ erlaubt Sekanten mit $x_2$ im Intervall von $x_2=0$ bis $x_2=1$.}
  \label{fig:limit:ddxf}
\end{figure}
\subsection{Allgemeinerer Grenzwertbegriff}

Der Grenzwertbegriff ist nicht nur auf Differenzialquotienten anwendbar, sondern auf beliebige reelle Funktionen $f(x)$.

\begin{Def}
  Eine reelle Funktion $f(x)$ mit einem reellen Argument $x$ strebt
  bei Annäherung von $x$ an eine Stelle $X$ gegen einen Grenzwert $F$,
  falls es für jede positive Fehlerschranke $\varepsilon>0$ einen
  Maximalabstand $\delta>0$ gibt, so dass für alle von $X$
  verschiedene Argumente $x$, die nicht weiter als $\delta$ von $X$
  entfernt sind, die Abweichung des Funktionswertes $f(x)$ vom
  Grenzwert $F$ nicht größer als die vorgegebene Fehlerschranke
  $\varepsilon$ ist, d.h., $|f(x)-F| \leq \varepsilon$ gilt.
\end{Def}
Als erstes Beispiel schauen wir uns den Grenzwert von
$f(x):=x\cdot \sin\l(\frac1x\r)$ bei Annäherung von $x$ an $0$
an. Aufgrund der Division durch $x$ ist $f(x)$ nicht an der Stelle
$x=0$ auswertbar.

Jedoch wird der Betrag von $\sin\l(\frac1x\r)$ nicht größer als
$1$. Damit ist der Betrag von $f(x)=x\cdot\sin\l(\frac1x\r)$ durch
$|x|$ beschränkt. Da $|x|$ bei Annäherung von $x$ an 0 den Grenzwert
$0$ hat, erwarten wir dass bei $|f(x)|$ auch der Fall ist.  Geben wir
uns eine beliebig knapp über 0 liegende Fehlerschranke $\eps$ vor und
untersuchen, für welchen Maximalabstand $\delta$ wir absichern können,
dass die Abweichung $|f(x)|$ nicht größer als $\eps$ wird.
\begin{align*}
  |f(x)-F| &= \l|x\cdot\sin\l(\frac1x\r) - 0\r|\\
           &\leq|x|\l|\sin\l(\frac1x\r)\r|\\
           &\leq |x|
\end{align*}
Wir können also die maximale Abweichung $\delta$ für $|x|$ gleich der
vorgegebenen Fehlerschranke $\eps$ wählen, damit diese nicht von
$\l|x\cdot\sin\l(\frac1x\r)\r|$ überschritten wird. Im linken Teil der
Abbildung~\ref{fig:limit:sin1x} ist als Beispiel die Fehlerschranke
$\eps=0.5$ vorgegeben und gezeigt, dass diese Fehlerschranke für
$x$-Werte im Intervall von $-0.5$ bis $0.5$ nicht überschritten wird.
\begin{figure}[H]
  \centering
  \psset{unit=3cm}
  \begin{pspicture}(-1.25,-1.25)(1.25,1.25)
    % \psgrid
    \psframe[linestyle=none,fillstyle=solid,fillcolor=lightblue](-0.5,-0.5)(0.5,0.5)
    \psaxes[Dx=0.5,Dy=0.5]{->}(0,0)(-1.25,-1.25)(1.25,1.5)[$x$,270][$y$,180]
    \psline[linecolor=gray](-1,-1)(1,1)
    \psline[linecolor=gray](-1,1)(1,-1)
    \def\myplot#1{
      \parametricplot[plotpoints=1000,linecolor=red]{1}{100.0}{
        1.0 t #1 div dup
        t #1 57.29 mul sin
        mul}}
    \myplot{}
    \myplot{neg}
    \rput[lb](-0.5,0.6){$\eps=0.5$}
    \rput[lt](0.55,-0.25){$\delta=0.5$}
  \end{pspicture}\quad
  \begin{pspicture}(-1.25,-1.25)(1.25,1.25)
    \psframe[linestyle=none,fillstyle=solid,fillcolor=lightblue](-1,-0.5)(1,0.5)
    \psaxes{->}(0,0)(-1.25,-1.25)(1.25,1.5)
    \def\myplot#1{
      \parametricplot[plotpoints=1000,linecolor=red]{1}{100.0}{
        1.0 t #1 div
        t #1 57.29 mul sin
      }}
    \myplot{}
    \myplot{neg}
    \rput[lb](-1,0.55){$\eps=0.5$}
  \end{pspicture}
  \caption{Links: Graph von $f(x)=x\cdot\sin\l(\frac1x\r)$; Rechts: Graph von $f(x)=\sin\l(\frac1x\r)$}
  \label{fig:limit:sin1x}
  % calc: 180/pi
  % ans:57.2957795131
\end{figure}
Als zweites Beispiel schauen wir uns die Funktion
$f(x)=\sin\l(\frac1x\r)$ an, die ebenfalls bei 0 nicht auswertbar ist.

Egal wie knapp über Null wir die maximale Abweichung $\delta$ wählen,
in dem Intervall von 0 bis $\delta$ liegen immer noch unendlich viele
Minimalstellen $\check x$ mit $\sin\l(1/\check x\r)=-1$ und unendlich
viele Maximalstellen $\hat x$ mit $\sin\l(1/\hat x\r)$.  Zur
Konstruktion jeweils einer dieser Stellen wählt einfach eine
hinreichend große natürliche Zahl $n$ mit $\frac1{2\pi n} \leq \delta$
und nutzt $\check x := \frac1{2\pi n +3\pi/4}$ beziehungsweise
$\hat x := \frac1{2\pi n +\pi/4}$. Alle größeren natürlichen Zahlen
$m>n$ liefern mit der selben Konstruktion weitere Minimalstellen und
Maximalstellen für $\sin(1/x)$ mit Abstand von Null, der kleiner als
der vorgegebene Abstand $\delta$ ist.

Jeder Versuch einen Grenzwert $F$ für $f$ bei Annäherung von $x$ an 0
zu konstruieren scheitert, denn diese Zahl $F$ hätte entweder bei
einem lokalen Minimum oder bei einem lokalen Maximum einen Abstand von mindestens $0.5$.

\subsection{Grenzwertsätze}
\label{sec:limit:theorems}
Seien $\alpha$ und $\beta$ zwei reelle Konstanten.  Konvergieren zwei
Funktionen $f(x)$ und $g(x)$ bei Annäherung von $x$ an eine Stelle $X$
gegen Grenzwerte $F$ beziehungsweise $G$, so konvergieren auch
$f(x)\cdot g(x)$ und $\alpha\cdot f(x)+\beta\cdot g(x)$ gegen
$F\cdot G$ beziehungsweise $\alpha F +\beta G$.

\paragraph{Produkt}
Konvergieren zwei Funktionen $f(x)$ und $g(x)$ bei Annäherung von $x$
an eine Stelle $X$ gegen Grenzwerte $F$ beziehungsweise $G$, so
konvergiert auch $f(x)\cdot g(x)$ gegen $F\cdot G$ beziehungsweise
$\alpha F +\beta G$.

Wir geben uns zunächst eine Fehlerschranke $\bar\eps>0$ für $f$ und
$g$ als Variable vor.  In Abhängigkeit von $\bar\eps$ finden wir eine
Formel für eine Schranke $\eps$ der Abweichung des Produkts $f(x)g(x)$
vom Grenzwertprodukt $FG$. Wir stellen diese Formel dann nach
$\bar\eps$ um, so dass wir $\eps$ vorgeben können, daraus $\bar\eps$
berechnen können und mit $\bar\eps$ dann den maximalen Abstand
$\delta$ ermitteln.

Um die Voraussetzungen $|f(x)-F|\leq\bar\eps$ und
$|g(x)-G|\leq\bar\eps$ anwenden zu können, addieren wir in
$|f(x)g(x)-FG|$ den Term $f(x)G$ und ziehen ihn gleich wieder ab. Das
ändert nichts am Wert unter dem Betragszeichen.
\begin{align*}
  \l|f(x) g(x) - FG\r| &= \l|f(x) g(x) - f(x)G + f(x)G - FG\r|
\end{align*}
Durch Ausklammern von $f(x)$ und $G$ gewinnen wir die Terme $f(x)-F$ und $g(x)-G$:
\begin{align*}
  &=\bigl|f(x)\cdot(g(x) - G) + (f(x)-F)\cdot G\bigr|
\end{align*}
Jetzt verwenden wir die für Summen allgemeingültige Abschätzung
$|a+b| \leq |a|+|b|$ um den Betrag der Summe in eine Summe von
Beträgen aufzusplitten.
\begin{align*}
  &\leq \l|f(x)\cdot(g(x) - G) \r| + \l|(f(x) - F)\cdot G\r|
\end{align*}
Bei Beträgen von Produkten ist es egal, ob man zuerst multipliziert
oder zuerst die Beträge bildet, d.h., es gilt allgemein
$|a\cdot b|=|a|\cdot |b|$, was wir auch zur Umformung der letzten Formel ausnutzen.
\begin{align*}
  &= |f(x)|\cdot |g(x)-G|+ \l|f(x) - F\r|\cdot|G|
\end{align*}
Bis auf $|f(x)|$ haben wir durch den bekannten Wert $G$ und die
vorgegebenen Fehlertoleranzen für $|g(x)-G|$ und $|f(x)-F|$ für alle
Beträge in der letzten Formel Abschätzungen.  Um auch $|f(x)|$
abschätzen zu können, fügen wir unter dem Betragszeichen $-F+F$
hinzu. In dem entstehenden Term $|f(x)-F+F|$ können wir $|f(x)-F|$ und
$|F|$ abschätzen.
\begin{align*}
  &= |f(x)-F+F|\cdot |g(x) - G | + |f(x)-F|\cdot |G|\\
  &\leq (\bar\eps + |F|)\cdot\bar\eps + \bar\eps |G|=:\eps.
\end{align*}
Letztendlich erhalten wir eine Gleichung mit der wir aus der
Fehlertoleranz $\bar\eps$ für $f$ und $g$ eine Fehlertoleranz $\eps$
für das Produkt $f\cdot g$ berechnen können.

Wir können auch $\eps$ vorgeben und das nötige $\bar\eps$ daraus
berechnen, wenn wir die Gleichung nach $\bar\eps$ auflösen:
\begin{align*}
  {\bar\eps}^2 + \l(|F|+|G|\r)\bar\eps -\eps &= 0\\
  \bar\eps&= -\frac{|F|+|G|}2 + \sqrt{\l(\frac{|F|+|G|}2\r)^2 + \eps}
\end{align*}
Es kommt nur die positive Wurzel als Lösung infrage, da die negative
Wurzel zu einer negativen Lösung führt, die sich nicht als
Fehlertoleranz eignet.

Für $\bar\eps$ gibt es Maximalabstände $\delta_f$ und $\delta_g$,
so dass Toleranzschranken $|f(x)-F|\leq \eps_f$ und
$|g(x)-G|\leq \eps_g$ eingehalten werden.

Wir nutzen den kleineren Maximalabstand
$\delta:=\min(\delta_f,\delta_g)$ um die zugehörige Fehlerschranken
für $|f(x)-F|\leq \bar\eps$, $|g(x)-G|\leq\bar\eps$ und somit auch für
$|f(x)g(x)-FG|\leq\eps$ zu erfüllen.

Zusammenfassend gesagt, finden wir also zu einer vorgegebenen
Fehlerschranke $\eps$ eine Maximalabweichung $\delta$, so dass für
alle Argumente $x$ mit $|x-X|\leq \delta$ die Ungleichung
$|f(x)g(x)-FG|\leq\eps$ erfüllt ist. Das Produkt $f(x)g(x)$ hat also
für die Annäherung von $x$ an $X$ den Grenzwert $FG$.

\paragraph{Linearkombination}
Seien $\alpha$ und $\beta$ zwei reelle Konstanten.  Konvergieren zwei
Funktionen $f(x)$ und $g(x)$ bei Annäherung von $x$ an eine Stelle $X$
gegen Grenzwerte $F$ beziehungsweise $G$, so konvergiert auch
$\alpha\cdot f(x)+\beta\cdot g(x)$ gegen den Grenzwert $\alpha\cdot F+\beta\cdot G$.

Für den uninteressanten Fall $\alpha=\beta=0$ ist klar, dass der
Grenzwert von $\alpha f(x) + \beta g(x)$ Null ist.  Nehmen wir jetzt
also an, dass mindestens eine der Zahlen $\alpha$ und $\beta$ von Null
verschieden ist.
\begin{align*}
  |\alpha f(x) + \beta g(x) - (\alpha F + \beta G)| &= |\alpha (f(x)-F) + \beta (g(x)-G)|\\
                                                    &\leq |\alpha|\cdot |f(x)-F| + |\beta|\cdot |g(x) - G|\\
                                                    &\leq (|\alpha|+|\beta|)\bar\eps=:\eps
\end{align*}
Da wir den Trivialfall $\alpha=\beta=0$ ausgeschlossen haben, können
wir nach $\bar\eps$ auflösen:
\begin{align*}
  \bar\eps &= \frac{\eps}{|\alpha|+|\beta|}
\end{align*}
Die Fehlertoleranz $\eps$ für die Linearkombination lässt sich also in
eine Fehlertoleranz $\bar\eps$ für $f$ und $g$ rückrechnen.  Damit
finden wir Maximalabweichungen $\delta_f$ und $\delta_g$, für $f$
beziehungsweise $g$, bei denen die Fehlertoleranz $\bar\eps$ jeweils
eingehalten wird.  Mit der Maximalabweichung
$\delta:=\min(\delta_f,\delta_g)$ wird für $f$ und $g$ die
Fehlertoleranz $\bar\eps$ eingehalten.  Die Linearkombination erfüllt
nach obiger Rechnung die Fehlertoleranz $\eps$. Der Grenzwert der
Linearkombination ist somit die Linearkombination der Grenzwerte $F$ und $G$.
\subsection{Stetigkeit}
\label{sec:limit:continuity}
Für den Differenzenquotient $\DDx f(x_1,x_2)$ der quadratischen
Funktion $f(x)=x^2$ haben wir in Abschnitt~\ref{sec:limit:ddx} die
nennerfreie Berechnungsvorschrift $\DDx f(x_1,x_2)=x_2+x_1$ gefunden
(siehe Gleichung~\eqref{eq:limit:DDxf}).  Die Auswertung der rechten
Seite an der Stelle $x_2=x_1$ hat uns zur Vermutung geführt, dass für
die Ableitung $\ddx f(x_1) = 2x_1$ gilt.

In diesem Abschnitt sehen wir, dass $g(x_2)=x_2+x_1$ ein Beispiel für
eine an der Stelle $x_1$ stetige Funktion. Bei diesen Funktionen kann
man den Grenzwert an der Stelle $x_2=x_1$ einfach durch Einsetzen von
$x_1$ berechnen.

\begin{Def}
  Sei $g(x)$ eine reelle Funktion in Abhängigkeit eines reellen
  Arguments $x$. Die Funktion~$g(x)$ ist an einer Stelle $X$ \emph{stetig},
  wenn der Grenzwert von $g(x)$ bei Annäherung von $x$ an $X$ gleich
  $g(X)$ ist.

  Ist $g(x)$ an allen Stellen $X$ stetig, so wird $g(x)$ einfach nur
  als stetig bezeichnet.
\end{Def}

Lineare Funktionen
\begin{align*}
  f(x) = p_0 + p_1 x
\end{align*}
sind stetig. Um uns davon zu überzeugen müssen wir nach der
Grenzwertdefinition zu jeder Fehlerschranke $\eps$ einen
Maximalabstand $\delta$ mit $|f(x)-f(X)|\leq \eps$ für alle $x$ mit
$|x-X|\leq \delta$ finden.
Ist $p_1$ gleich null, so ist der Fehler $|f(x)-f(X)|$ unabhängig von $x$ und $X$ gleich null, der Grenzwert ist also $f(X)$.

Für den Fall $p_1\neq 0$ nutzen wir $\delta =\frac{\eps}{|p_1|}$ als Maximalabstand für den die Fehlerschranke $\eps$ eingehalten wird:
\begin{align*}
  |f(x)-f(X)| = |p_0+p_1 x - (p_0+p_1X)| &= |p_1|\cdot|x-X|\leq |p_1|\frac{\eps}{|p_1|} = \eps
\end{align*}
Lineare Funktionen sind also stetig.

Nach Abschnitt~\ref{sec:limit:theorems} ist der Grenzwert eines Produkts
gleich dem Produkt der Grenzwerte ist und der Grenzwert einer
Linearkombination gleich der Linearkombination der Grenzwerte.

Die Stetigkeit der Potenzfunktionen $f(x)=x^n$ für $n=0,1,2,\ldots$
ergibt sich daraus, dass diese als Produkte der linearen Funktion
$g(x)=x$ mit sich selber darstellbar sind.

Polynomfunktionen $p(x) = p_0+p_1\cdot x + \ldots + p_n x^n$ sind als
Linearkombinationen von Potenzfunktionen ebenfalls stetig.

\section{Polynome}
\subsection{Monome und Polynome}
\label{sec:poly}
Polynome sind wichtige Hilfsmittel in der Mathematik und Physik. Zum
Beispiel kann man mit ihnen an Stützstellen vorgegebene Werte glatt
interpolieren und Differentialgleichungen der Sorte, die wir im
Einleitungsabschnitt kennengelernt haben, numerisch lösen.

Die Potenzfunktionen $f(x)=x^n$ mit $n=1,2,\ldots$ werden auch als
\emph{Monome} bezeichnet. Linearkombination
\begin{align}
  f(x) &= p_0 + p_1\cdot x + p_2\cdot x^2 + \ldots + p_n\cdot x^n
         \label{eq:poly}
\end{align}
von Monomen sind \emph{Polynome}. Die Koeffizienten $p_i$ mit $i=0,\ldots,n$ sind dabei vorgegebene Konstanten.

Für Summen, wie in Gleichung~\eqref{eq:poly} nutzen wir im Folgenden auch das Summenzeichen
\begin{align}
  f(x) &= \sum_{i=0}^n p_i x^i
         \label{eq:poly:sum}
\end{align}
Obwohl $x^0$ an der Stelle $x=0$ nicht auswertbar ist, wird formal
vereinbart, dass bei Variablen wie $x$ der Term $x^0$ als $1$ zu
interpretieren ist, um die Kurzschreibweise~\eqref{eq:poly:sum}
effizient einsetzen zu können.

Polynome ersten Grades, d.h., mit $n=1$ kennst du sicher, das sind
linearen Funktionen mit Konstantanteil $p_0$ und Anstieg $p_1$.  Zum
Beispiel ergibt sich mit den Koeffizienten $p_0=1$ und $p_1=2$ das
Polynom $p(x)=1+2x$, das im nächsten Abschnitt als
Gleichung~\eqref{eq:interpol:px01} auftaucht.

Auch quadratische Polynome, d.h., mit $n=2$ habt ihr sicher schon in
der Schule gehabt.  Beispielsweise ist $p(x)=1 + \frac 72x-\frac32x^2$
das im nächsten Abschnitt berechnete
Interpolationspolynom~\eqref{eq:interpol:px012} mit $p_0=1$,
$p_1=\frac72$ und $p_2=-\frac32$.
\subsection{Polynominterpolation}
\label{sec:interpol}
Zur Erstellung eines Höhenprofils eines Berges wird an einer Reihe von
Stellen $x_0,x_1,\ldots,x_n$ in horizontaler Richtung jeweils die Höhe $y_0,y_1,\ldots,y_n$
vermessen. Man hat nur einzelne Stellen und einzelne Höhenwerte, der
Höhenverlauf des Berges ist jedoch eigentlich glatt.

Verbindet man die Punkte $(x_0,y_0), (x_1,y_1),\ldots,(x_n,y_n)$
durch Geradenstücke, so entstehen an den Verbindungspunkten
$(x_k,y_k)$ mit $k=1,\ldots,n-1$ unerwünschte Knicke (siehe 
verbindender Polygonzug in Abbildung~\ref{fig:interpol:example}).

Statt der Verbindungsgeraden kann man auch eine glatte Funktion $p(x)$
zur Verbindung der Punkte nutzen. Als Bedingung dafür, dass diese Funktion die Punkte
$(x_k,y_k)$ mit $k=0,\ldots,n$ verbindet, muss $p(x)$ an den Stellen $x_k$ die Werte $y_k$ annehmen:
\begin{align*}
  p(x_k) = y_k
\end{align*}
für $k=0,\ldots,n$. Hat man die Funktion $p(x)$
konstruiert, kann man sie auch an anderen Stellen als
$x=x_0,\ldots,x_n$ auswerten. Die Konstruktion der Funktion $p(x)$ und
die Auswertung zwischen den vorgegebenen Stellen bezeichnet man als \emph{Interpolation}.

Am Häufigsten nutzt man Polynome als glatte Funktion $p(x)$ bei der
Interpolation.

Wir schauen uns im Folgenden das leicht verständliche und wichtige
Neville-Verfahren zur Polynominterpolation an. Dieses Verfahren bildet
zum Beispiel auch die Grundlage für die bekanntere
Newton-Interpolation, auf die wir hier jedoch nicht eingehen.

Beim Neville-Verfahren baut man das Interpolationspolynom
rekursiv auf.

Sei $p(x;x_0,\ldots,x_{n-1})$ ein Polynom mit Höchstgrad $(n-1)$, das
die Werte $y_0,\ldots,y_{n-1}$ an den Stellen $x_0,\ldots,x_{n-1}$
interpoliert und analog sei $p(x;x_1,\ldots,x_n)$ ein Polynom mit
Höchstgrad $(n-1)$, das die Werte $y_1,\ldots,y_n$ an den Stellen
$x_1,\ldots,x_n$ interpoliert.

Wir schauen uns die Eigenschaften des daraus konstruierten Polynoms
\begin{align}
  p(x;x_0,\ldots,x_n) &:= \frac{p(x;x_0,\ldots,x_{n-1})\cdot(x_n-x) + p(x;x_1,\ldots,x_n)\cdot(x-x_0)}{x_n-x_0}.
                        \label{eq:interpol:Neville}
\end{align}
an:
\begin{itemize}
\item Das konstruierte Polynom hat Höchstgrad $n$, denn im Zähler
  kommen die Faktoren $(x_n-x)$ und $(x-x_0)$ hinzu, in denen jeweils
  einmal $x$ vorkommt. Es kann jedoch sein, dass sich die entstehenden
  Terme von Grad $n$ gerade aufheben. Deshalb kann man nur eine Aussage für
  den Höchstgrad treffen und nicht für den Polynomgrad selber.
\item Das konstruierte Polynom interpoliert die Werte $y_1,\ldots,y_{n-1}$ an den Stellen
  $x_1,\ldots,x_{n-1}$:
  
  Für $k=1,\ldots,n-1$ haben
  $p(x_k;x_0,\ldots,x_{n-1})$ und $p(x_k;x_1,\ldots,x_{n})$ den
  gleichen Wert $y_k$ und man kann diesen Faktor ausklammern
  \begin{align*}
    p(x_i;x_0,\ldots,x_n) &= \frac{y_k\cdot((x_n-x) + (x-x_0))}{x_n-x_0} = y_k
  \end{align*}
\item Zusätzlich interpoliert das konstruierte Polynom den Wert $y_0$
  an der Stelle $x_0$ und den Wert $y_n$ an der Stelle $x_n$:
  
  Bei $x=x_0$ fällt der hintere Term des Zählers mit Faktor $(x-x_0)$ weg. Übrig bleibt
  \begin{align*}
    p(x_0;x_0,\ldots,x_n)&=\frac{p(x_0,x_0,\ldots,x_{n-1})\cdot(x_n-x_0)}{x_n-x_0} = p(x_0,x_0,\ldots,x_{n-1}) = y_0
  \end{align*}
  Dabei wurde berücksichtigt, dass $p(x;x_0,\ldots,x_{n-1})$ an der
  Stelle $x_0$ den Wert $y_0$ interpoliert.
  
  Analog fällt bei $x=x_n$ der vordere Term mit Faktor $(x_n-x)$ weg.
  Es ergibt sich wie im vorhergehenden Fall $p(x_n;x_0,\ldots,x_n)=p(x_n;x_1,\ldots,x_n)=y_n$.
\end{itemize}

Als Start für die rekursive Konstruktion der Interpolationspolynome
mittels~\eqref{eq:interpol:Neville} kann man die konstanten Polynome $p(x,x_0)=y_0$ nutzen.

\paragraph{Beispiel}
Wir konstruieren das Interpolationspolynom 2. Grades mit folgenden Interpolationspunkten:
\begin{table}[H]
  \centering
  \begin{tabular}[H]{l|l|l}
    $i$ & $x_i$ & $y_i$ \\\hline
    0 & 0 & 1 \\
    1 & 1 & 3 \\
    2 & 2 & 2
  \end{tabular}
  \caption{Stützstellen und Werte für das Interpolationspolynom}
\end{table}
Polynome vom Grad 0, die die Daten an jeweils einer Stelle interpolieren:
\begin{align*}
  p(x;0) &= 1&  p(x;1) &= 3&  p(x;2)&=2
\end{align*}
Polynome vom Grad 1, die die Daten an jeweils zwei Stellen interpolieren:
\begin{align}
  p(x;0,1) &= \frac{p(x;0)\cdot(x_1-x) + p(x;1)\cdot(x-x_0)}{x_1-x_0}
             = \frac{1\cdot(1-x) + 3\cdot(x-0)}{1-0}
             = 2x+1\label{eq:interpol:px01}\\
  p(x;1,2) &= \frac{p(x;1)\cdot(x_2-x) + p(x;2)\cdot(x-x_1)}{x_2-x_1}
             = \frac{3\cdot (2-x) + 2\cdot(x-1)}{2-1}
             = -x+4\label{eq:interpol:px12}
\end{align}
Polynom vom Grad 2, das den gesamten Datensatz interpoliert:
\begin{align}
  p(x;0,1,2) &= \frac{p(x;x_0,x_1)\cdot(x_2-x) + p(x;x_1,x_2)\cdot (x-x_0)}{x_2-x_0}\notag\\
             &= \frac{(2x+1)\cdot(2-x) + (-x+4)\cdot(x-0)}{2-0}\notag\\
             &= -\frac32 x^2 +\frac72x +1\label{eq:interpol:px012}
\end{align}
Das resultierende Polynom ist in Abbildung~\ref{fig:interpol:example}
dargestellt. Die Interpolationspunkte an den Stellen $(x_i,y_i)$ sind
durch kleine blaue Punkte gekennzeichnet.
\begin{figure}[H]
  \centering
  \begin{pspicture}(-0.5,-0.5)(2,4)
    % \psgrid
    \psline[linecolor=green](0,1)(1,3)(2,2)
    \psaxes{->}(0,0)(0,0)(2.5,3.5)[$x$,270][$y$,180]
    \psplot[linecolor=red]{0}{2}{
      -1.5 x mul
      3.5 add x mul
      1 add
    }
    \psdots[linecolor=blue](0,1)(1,3)(2,2)
  \end{pspicture}
  \caption{Verbindender Polygonzug ({\color{green}grün}) und Interpolationspolynom
    ({\color{red}rot}) für die {\color{blue}blau} dargestellten Punkte $(x_0,y_0)=(0,1)$,
    $(x_1,y_1)=(1,3)$ und $(x_2,y_2)=(2,2)$}
  \label{fig:interpol:example}
\end{figure}
\subsection{Differenzialquotienten von Polynomen}
In Abschnitt~\ref{sec:poly} haben wir Polynome
\begin{align*}
  p(x) &= \sum_{i=0}^n p_i x^i
\end{align*}
als Linearkombinationen von Monomen $x^0, x^1,\ldots$ kennengelernt.
\subsection{Differenzialquotienten von linearen Funktionen und Monomen}
In der Einleitung wurde der Differenzialquotient als Anstieg der
Tangenten eingeführt. Lineare Funktionen beschreiben Geraden, bei
denen der Anstieg konstant ist.

Alle Sekanten einer linearen Funktion stimmen mit der durch sie
beschriebenen Gerade überein und haben den Anstieg dieser
Gerade. Als Grenzwert haben auch alle Tangenten diesen Anstieg.

Für den formalen Test beschreiben wir die lineare Funktion mit zwei
Konstanten $p_0$ und $p_1$ durch die Formel
\begin{align*}
  f(x) &= p_0 + p_1 x
\end{align*}
Für beliebige voneinander verschiedene Stellen $x_1,x_2$ ergibt sich
der Sekantenanstieg gemäß
\begin{align}
  \frac{\Delta f}{\Delta x}(x_1,x_2) &= \frac{f(x_2) - f(x_1)}{x_2-x_1}\notag\\
                                     &= \frac{(p_0+p_1x_2) - (p_0+p_1x_1)}{x_2-x_1}\notag\\
                                     &= \frac{p_1(x_2-x_1)}{x_2-x_1} = p_1.\label{eq:limit:ddxp1}
\end{align}
Der Anstieg aller Sekanten ist also konstant gleich $p_1$ und damit
ist auch der Anstieg als Grenzwert des Sekantenanstiegs gleich $p_1$.

Aus den Differenzenquotienten der Potenzfunktion $f(x)=x^n$ vom Grad $n=2,3,\ldots$
\begin{align*}
  \frac{\Delta f}{\Delta x}(x_1,x_2) &= \frac{f(x_2) - f(x_1)}{x_2-x_1}\\
                                     &= \frac{x_2^n - x_1^n}{x_2-x_1}
\end{align*}
erhält man mit der erweiterten binomischen Formel
$a^n-b^n=(a-b)\cdot\l(\sum_{i=0}^{n-1}a^ib^{n-1-i}\r)$ aus
Anhang~\ref{sec:extendedBinomial} wieder eine Form
\begin{align*}
  \DDx f(x_1,x_2)&= \frac{(x_2-x_1)\cdot\l(\sum_{i=0}^{n-1} x_2^ix_1^{n-1-i}\r)}{x_2-x_1},
\end{align*}
aus der sich der Nenner herauskürzt:
\begin{align*}
  \DDx f(x_1,x_2)&= \sum_{i=0}^{n-1} x_2^ix_1^{n-1-i}
\end{align*}
Die rechte Seite in der letzten Darstellung des Differenzenquotienten
ist an der Stelle $x_2=x_1$ stetig.  Wir können den
Differenzialquotient also durch Auswertung der rechten Seite an der
Stelle $x_2=x_1$ berechnen:
\begin{align}
  \ddx f(x_1) &= \sum_{i=0}^{n-1} \underbrace{x_1^i x_1^{n-1-i}}_{\text{$n$-mal $x_1^{n-1}$}} \\
  \ddx f(x) &= n x^{n-1}\label{eq:diff:monom}
\end{align}
Die letzte Gleichung ist eigentlich nur für $n=2,3,\ldots$ anwendbar,
da für $x=0$ in den Fällen $n=0$ und $n=1$ auf die nicht definierten
Terme $0\cdot\frac{1}{0}$ beziehungsweise $1\cdot 0^0$ führt.

Jedoch wird, wie bei der Einführung des Summenzeichens in
Abschnitt~\ref{sec:poly} zur Vereinfachung der Schreibweise zumindest
im Fall $n=1$ der Term $0^0$ als $1$ interpretiert, so dass die Formel
$\ddx f(x) = n x^{n-1}$ auch in diesem Fall gültig bleibt.
\subsection{Linearität von Differenzen- und Differenzialquotienten}
\label{sec:ddx:linearity}
Der Differenzialquotient ist als auf Funktionen anzuwendende Operation
\emph{linear}.  Diese Eigenschaft ist außerordentlich
wichtig. Abschnitt~\ref{sec:poly:ddx} wird uns einen Eindruck davon
vermitteln, wenn wir mit Hilfe der Linearität Differenzialquotienten
von Polynomen auf Differenzialquotienten von Monomen zurückführen.

In diesem Abschnitt werden wir uns zuerst anschauen, was die
Linearität des Differenzialquotienten bedeutet soll. Danach überzeugen
wir uns davon sich, dass Differenzenquotienten ebenfalls linear sind
und dass sich diese Eigenschaft durch Grenzwertbildung auf
Differenzialquotienten überträgt.

Seien $\alpha$, $\beta$ reelle Konstanten und $f(x)$, $g(x)$ reelle
Funktionen, die an der Stelle $x$ differenzierbar sind.

Dann ist die Linearkombination
\begin{align*}
  f(x) = \alpha g(x) + \beta h(x)
\end{align*}
der Funktionen ebenfalls differenzierbar und es gilt die Gleichung
\begin{align*}
  \ddx f(x) = \ddx{}(\alpha g(x) + \beta h(x)) &= \alpha \ddx g(x) + \beta\ddx h(x),
\end{align*}
das heißt, der Differenzialquotient ist \emph{linear}.

Analog gelten für Differenzenquotienten der Funktion $f$ mit zwei
voneinander verschiedenen Stellen $x_1,x_2$ die Gleichungen
\begin{align*}
  \frac{\Delta f}{\Delta x}(x_1,x_2) &= \frac{f(x_2) - f(x_1)}{x_2-x_1}\\
                            &= \frac{\alpha g(x_2) + \beta h(x_2) - (\alpha g(x_1) + \beta h(x_1))}{x_2-x_1}\\
                            &= \alpha \frac{g(x_2) - g(x_1)}{x_2-x_1} + \beta \frac{h(x_2)- h(x_1)}{x_2-x_1}\\
                            &= \alpha\frac{\Delta}{\Delta x} g(x_1,x_2) + \beta\frac{\Delta}{\Delta x} h(x_1,x_2)
\end{align*}
Im Abschnitt~\ref{sec:limit:theorems} über Grenzwertsätze haben wir
uns schon davon überzeugt, dass der Grenzwert einer Linearkombination
gleich der Linearkombination von Grenzwerten ist. Bilden wir auf beiden Seiten der Gleichung
\begin{align*}
  \DDx f(x_1,x_2) = \alpha\DDx g(x_1,x_2) + \beta\DDx f(x_1,x_2)
\end{align*}
den Grenzwert für die Annäherung von $x_2$ an $x_1$, so erhalten wir die Gleichung
\begin{align*}
  \ddx f(x_1) &= \alpha \ddx g(x_1) + \beta\ddx f(x_1). 
\end{align*}
Somit hat sich die Linearität des Differenzialquotienten bestätigt.
\subsection{Differenzialquotienten von Polynomen}
\label{sec:poly:ddx}
In Abschnitt~\ref{sec:poly} wurden Monome als Potenzfunktionen $x^n$
mit $n=0,1,2,\ldots$ eingeführt, wobei $x^0=1$ gesetzt wird unabhängig
davon dass $x$ auch null werden kann.
Polynome sind Linearkombinationen
\begin{align*}
  p(x)&=p_0+p_1x+\ldots+p_nx^n 
  \intertext{oder mit Summenschreibweise}
  p(x)&=\sum_{i=0}^n p_i x^i
\end{align*}
von Monomen mit konstanten Koeffizienten $p_0,p_1,\ldots,p_n$.

Die Ableitungen $\ddx{}\left(x^n\right) = n\cdot x^{n-1}$ für
$n=1,2,\ldots$ von Monomen haben wir in Abschnitt~\ref{sec:poly:ddx}
kennengelernt. Dabei interpretieren wir $\frac{0}{x}=0$ und $x^0 = 1$
unabhängig davon, dass $x$ auch null werden kann.

In Abschnitt~\ref{sec:ddx:linearity} haben wir uns davon überzeugt,
dass der Differenzenquotient eine lineare Operation darstellt, dass
also der Differenzialquotient einer Linearkombination
differenzierbarer Funktionen gleich der Linearkombination der
Differenzialquotienten der Funktionen ist.

Die Formel für die Ableitung von Polynomen ist jetzt nur noch ein
Zusammensetzen all dieser Bausteine.

Da die Monome differenzierbar sind und die Polynome Linearkombination
von Monomen, ist die Ableitung von Polynomen gerade die zugehörige
Linearkombination der Differenzialquotienten der eingehenden Monome.

Wir gehen von der Summenform eines Polynoms aus.
\begin{align*}
  p(x) = \sum_{i=0}^n p_i x^i
\end{align*}
De Differenzialquotient des Konstantanteils $p_0$ ist null. Das
berücksichtigen wir, indem wir die Summe für den Differenzialquotient
des Polynoms erst ab $1$ laufen lassen. Die Differenzialquotienten der
anderen Monome haben wir uns eben angesehen.
\begin{align*}
  \ddx p(x) = \sum_{i=1}^n p_i i x^{i-1}.
\end{align*}
\subsection{Numerische Lösung von Differenzialgleichungen}
\label{sec:numODE}
Zur Lösung von Differenzialgleichungen, wie wir sie im
Einleitungsabschnitt kennengelernt haben, gibt es viele numerische
Verfahren.

Ein Großteil dieser Verfahren nutzt Polynominterpolation oder
Polynomapproximation.

Wir lösen die Differenzialgleichung
\begin{align}
  \ddx y(x) &= \alpha y(x)
              \label{eq:numODE:example}
\end{align}
mit einem solchen Verfahren numerisch approximativ für kleine Beträge
von $x$. Die Differenzialgleichung~\eqref{eq:numODE:example} passt zur
Gleichung~\eqref{eq:intro:et:dgl} für das elektrische Netzwerk in
Abschnitt~\ref{sec:intro:et}, wenn wir $x=t$, $y=U$,
$\alpha=-\frac1\tau$ und $U_B=0$ setzten. Sie beschreibt dann den
Entladevorgang eines RC-Gliedes.

Wir nutzen bei dem numerischen Verfahren aus, dass sich der Betrag von
$x^n$ stark verkleinert, wenn der Betrag von $x$ kleiner wird. Nutzen
wir zum Beispiel $x$ nur im Bereich von $-0.1$ bis $0.1$, so ändert
sich zum Beispiel $x^5$ nur im Bereich von $-0.00001$ bis $-0.00001$.

Davon lassen wir uns leiten und setzen für $y(x)$ in Gleichung~\eqref{eq:numODE:example} den polynomialen Ansatz
\begin{align}
  y(x) &= \sum_{k=0}^n c_k x^k
         \label{eq:numODE:Poly}
\end{align}
ein. Wir gleichen dann die bisher noch freien Koeffizienten $c_k$ so
ab, dass die Gleichung für so viele Potenzen niedrigen Grades erfüllt
wird, wie es möglich ist. Den Fehler in den höheren Potenzen von $x$
vernachlässigen wir. Wir wissen ja bereits, dass dieser Fehler sehr
klein wird, wenn wir den Betrag von $x$ hinreichend klein wählen.

Einsetzen des Ansatzes~\eqref{eq:numODE:Poly} in Gleichung~\eqref{eq:numODE:example} liefert:
\begin{align*}
  \ddx{}\l(\sum_{k=0}^n c_k x^k\r) &=\alpha\l(\sum_{k=1}^n c_k x^k\r)
\end{align*}

Auf der linken Seite wenden wir die Differenziation auf das Polynom an:
\begin{align*}
  \sum_{k=1}^n c_k k x^{k-1} &= \sum_{k=0}^n  \alpha c_k x^k
\end{align*}
Hier sei noch einmal daran erinnert, dass beim Differenzieren der
erste Summand $c_0x^0=c_0$ von $\sum_{k=0}^n c_kx^k$
herausfällt. Deshalb fängt die Summe erst bei Index 1 an. Für einen
Vergleich der Koeffizienten zu den Potenzen von $x$ ist es günstiger,
wenn links und rechts die selben Potenzen von $x$ stehen. Deshalb
ersetzen wir auf der rechten Seite den Exponenten $k$ durch einen
neuen Index $\bar k-1$. Aus $\bar k-1 = k$ ergibt sich $\bar k=k+1$.
\begin{align*}
  \sum_{k = 1}^{n} c_k k x^{k-1} = \sum_{\bar k=1}^{n+1}  \alpha c_{\bar k-1} x^{\bar k-1}
\end{align*}
Die Zählvariable $\bar k$ ist der Summationsindex der Summe auf der
rechten Seite, für den wir uns den Namen selber aussuchen können,
solange er nicht mit bereits in der Summe vorkommenden Variablennamen
kollidiert. Wir bezeichnen ihn wieder mit $k$, damit der Vergleich der
in beiden Summen auftretenden Potenzen von $x$ einfacher wird.

Bei der entstehenden Gleichung
\begin{align}
  \sum_{k = 1}^{n} c_k k x^{k-1} = \sum_{k=1}^{n+1}  \alpha c_{k-1} x^{k-1}
  \label{eq:numODE:poly}
\end{align}
können wir die Koeffizienten der Potenzen $x^{k-1}$ für $k=1,\ldots,n$
abgleichen. Die Koeffizienten zu $x^{k-1}$ für $k=1,\ldots,n$ auf der
linken und rechten Seite der Gleichung sind $c_k k$ beziehungsweise
$\alpha c_{k-1}$. Die Gleichungen für deren Abgleich lauten
\begin{align*}
  k c_{k} &= \alpha c_{k-1}.
\end{align*}
Der Koeffizientenabgleich für die höchste Potenz $x^n$ in
Gleichung~\eqref{eq:numODE:poly} liefert die Gleichung
\begin{align*}
  0 &= \alpha c_n
\end{align*}
Damit erhält man das Gleichungssystem
\begin{align}
  1\cdot c_1 &= \alpha c_0\label{eq:numODE:x0}\\
  2\cdot c_2 &= \alpha c_1\\
             &\vdots\\
  n\cdot c_{n} &= \alpha c_{n-1}\label{eq:numODE:xn1}\\
  0 &= \alpha c_n\label{eq:numODE:xn}
\end{align}
für die Koeffizienten $c_k$.

Dieses Gleichungssystem ist von unten nach oben lösbar
($c_n=0 \Rightarrow c_{n-1}=0 \Rightarrow \ldots\Rightarrow c_0 = 0$)
und hat nur noch die Triviallösung, dass alle Koeffizienten $c_k$ null
sind, also $y(x)=0$ gilt.

Um nichttriviale Lösungen zu erhalten, vernachlässigen wir den
Koeffizientenvergleich für die höchste Potenz.

Das kann man zum Beispiel so interpretieren, dass man statt der Lösung
für \eqref{eq:numODE:example} die exakte Lösung der leicht gestörten
Differenzialgleichung
\begin{align}
  \tau \ddx y &= \alpha y - \alpha c_n x^n
                \label{eq:numODE:poly:mod}
\end{align}
ermittelt. Die Störung der Differenzialgleichung bleibt klein, wenn
wir uns nicht zu weit hinaus wagen, das heißt, wenn wir die
Differenzialgleichung nur für kleine Werte von $x$ nutzen.

Der Koeffizientenvergleich für den
Polynomansatz~\eqref{eq:numODE:Poly} bei~\eqref{eq:numODE:poly:mod}
liefert anstelle von~\eqref{eq:numODE:xn} die immer erfüllte Gleichung
$0=0$, die aus dem System gestrichen werden kann.

Es bleiben die Gleichungen
\begin{align*}
  k c_k &= \alpha c_{k-1}
\end{align*}
für $k=1,\ldots,n$, oder umgestellt nach $c_k$ die Gleichungen
\begin{align}
  c_k = \alpha\frac{c_{k-1}}k.
  \label{eq:numODE:coeffs}
\end{align}
In der letzten Form kann man die Gleichungen als rekursive
Zuweisungskette interpretieren. Der Koeffizient $c_k$ wird aus dem
vorhergehenden $c_{k-1}$ berechnet. durch Multiplikation mit $\alpha$ und
Division durch $k$ berechnet.

Der Koeffizient $c_0$ bleibt dabei frei wählbar. Aus~\eqref{eq:numODE:Poly} erkennt man, dass $c_0$ gerade der Wert von $y(x)$ für $x=0$ ist. (Im Beispiel unseres Entladevorgangs ist das die Anfangsspannung des Kondensators.)

Wir können also zusätzlich 
\begin{align*}
  c_0 &= y(0)
\end{align*}
schreiben.

Bei jedem Schritt $k=1,\ldots,n$ des Systems~\eqref{eq:numODE:coeffs} von Zuweisungen kommt ein Faktor $\alpha$ hinzu und es wird durch das aktuelle $k$ dividiert. So erhält man für die Koeffizienten die Berechnungsvorschrift
\begin{align*}
  c_k &= \frac{\alpha^k y(0)}{k\cdot(k-1)\cdot\ldots\cdot 2\cdot 1}.
\end{align*}
\begin{align*}
  y(x) &\approx y(0)\cdot\sum_{k=0}^{n} \frac{1}{k!} (\alpha x)^k
\end{align*}
\begin{figure}[H]
  \centering
  \begin{pspicture}(-0.5,-8)(10.5,5.5)
    %\psgrid
    \def\myRef{2.71828 x neg exp }
    \def\myApprox{
      % transforms the 3 last elements on stack
      % stack: partialSum summand iter
      /myRecursion
      {
        1 add % update counter
        dup % stack: partialSum summand iter iter
        % ( <1> ) print stack
        3 1 roll % stack: partialSum iter summand iter
        % construction of the new summand: summand := summand/iter * alpha * x
        % ( <2> ) print stack
        div
        \myAlpha mul
        x mul
        dup % stack: partialSum iter summand summand
        % ( <3> ) print stack
        4 -1 roll % iter summand summand partialSum
        % ( <4> ) print stack
        add % stack: iter summand partialSum
        exch % stack: iter partialSum summand
        % ( <5> ) print stack
        3 -1 roll
        % ( <6> ) print stack
      } def
      % Start values:
      1.0 % partialSum
      1.0 % summand
      0 % iter
      \myDegree { myRecursion } repeat
      pop pop 
    }
    \def\myAlpha{-1.0 }    
    \def\myDegree{8 }
    \rput[bl](0,0){\begin{psgraph}{->}(0,0)(4.5,1.5){10cm}{5cm}
        \psplot[linecolor=green]{0}{4}{
          \myRef
        }
        \psplot[linecolor=blue]{0}{4}{
          \myApprox
        }
        \def\myDegree{10 }
        \psplot[linecolor=red]{0}{4}{
          \myApprox
        }
      \end{psgraph}}
    \def\myErr{
      \myRef \myApprox sub
      abs
      dup 1e-9 le { pop 1e-7 } if
      log
    }
    \rput[tl](0,-1){\begin{psgraph}[Oy=-5,ylogBase=10]{->}(0,-5)(0,-7.5)(4.5,1.0){10cm}{7cm}
        \def\myDegree{8 }
        \psplot[linecolor=blue]{0}{4}{
          \myErr
        }
        \def\myDegree{10 }
        \psplot[linecolor=red]{0}{4}{
          \myErr
        }
      \end{psgraph}}
  \end{pspicture}
  \caption{Approximation der Differenzialgleichung $\ddt y = -y$}
  \label{fig:poly:exp}
\end{figure}
\appendix
\section{Erweiterte Binomische Formel}
\label{sec:extendedBinomial}
\begin{align*}
  (a-b)\cdot\l(\sum_{i=0}^{n-1} a^ib^{n-1-i}\r)
  &= \sum_{i=0}^{n-1} a^{i+1}b^{n-1-i} - \sum_{i=0}^{n-1} a^i b^{n-i}\\
\end{align*}
$\bar i = i+1$
\begin{align*}
  &= \sum_{\bar i = 1}^n a^{\bar i} b^{n-\bar i} - \sum_{i=0}^{n-1} a^i b^{n-i}\\
  &= \underbrace{a^n}_{\bar i=n} + \underbrace{\l(\sum_{i=1}^{n-1} a^i b^{n-i} - a^i b^{n-i}\r)}_{=0} - \underbrace{b^n}_{i=0}\\
  &= a^n - b^n
\end{align*}
\end{document}

%%% Local Variables:
%%% eval: (flyspell "de_DE")
%%% End: